\documentclass[a4paper,11pt]{article}

% Set page size and margins
\usepackage[a4paper,top=2cm,bottom=2cm,left=3cm,right=3cm,marginparwidth=1.75cm]{geometry}

%Karnaugh Maps
\usepackage{karnaugh-map}

% TOC
\usepackage{titlesec}

% Lists
\usepackage{enumitem}

% Tables
\usepackage{tabularx}

% Code rendering
\usepackage{listings}

% Useful packages
\usepackage{amsmath}
\usepackage{mathtools}
\usepackage{graphicx}
\usepackage[colorlinks=true,linkcolor={black},citecolor={blue!80!black},urlcolor={blue!80!black}]{hyperref}

% Custom shorthands
%% Table cmds
\newcommand\HSPC{\rule{0pt}{4ex}\rule[-2.0ex]{0pt}{0pt}}
\newcommand\RSPC{\rule{0pt}{2.5ex}\rule[-1.25ex]{0pt}{0pt}}

%% Math cmds
\newcommand{\mathfsc}[2]{#1#2\normalsize}

% Code bloc styling
\lstdefinestyle{myStyle}{
    breaklines=true,
    frame=single,
    numbers=left,
    numbersep=10pt,
    numberstyle=\normalsize\color{gray!80!black},
    basicstyle=\large\ttfamily,
    keywordstyle=\bfseries\color[HTML]{FE4A49},
    commentstyle=\itshape\color[HTML]{A6A6A8},
    identifierstyle=\color[HTML]{2AB7CA},
    backgroundcolor=\color[HTML]{F4F4F8},
    stringstyle=\color[HTML]{FED766}
}


\title{\textbf{Evaluation of Minterm Expression}}
\author{Gaurav Prajapati}

\begin{document}
\date{}
\maketitle

% Make Table of contents from Sections
\tableofcontents

% Problem Statement
\newpage
\section{Problem}
\label{sec:ques}
\textbf{(GATE CS-2014 SET-3)}\\
\textbf{Q.7} Consider the Following Minterm expression for F:
    \mathfsc{\large}{\[\mathtt{F (P,Q,R,S) = \sum 0,2,5,7,8,10,13,15}\]}
The minterms 2,7,8 and 13 are \textit{'do not care'} terms. The minimal sum-of-products form for F is
\begin{enumerate}[label=(\Alph*)]
    \item \mathfsc{\large}{\(\mathtt{Q\Bar{S} + \Bar{Q}S}\)}
    \item \mathfsc{\large}{\(\mathtt{\Bar{Q}\Bar{S} + QS}\)}
    \item \mathfsc{\large}{\(\mathtt{\Bar{Q}\Bar{R}\Bar{S} + \Bar{Q}R\Bar{S} + Q\Bar{R}S + QRS}\)}
    \item \mathfsc{\large}{\(\mathtt{\Bar{P}\Bar{Q}\Bar{S} + \Bar{P}QS + PQS + P\Bar{Q}\Bar{S}}\)}
\end{enumerate}
\bigskip

\section{Introduction}
\paragraph{}
For a given set of Boolean Logic Inputs, we can define the following terms:
\begin{itemize}
    \item \textbf{Minterm} is a boolean expression resulting in an output of 1 for the minimum number of cells in a Karnaugh-Map (K-Map) and 0 in other cells.
    \item \textbf{Sum of Products} is a boolean expression for the \textit{Sum} (OR) of various \textit{Product} (AND) terms.
    \item \textbf{'do not care'} terms for a boolean expression are the set of input values for which the output of the function does not matter. The value for these can be taken as 0 or 1 by choice
\end{itemize}
\bigskip

% Components used
\section{Components}
\begin{table}[ht]
\centering
\begin{tabularx}{0.8\textwidth} {
    | >{\raggedright\arraybackslash}X
    | >{\centering\arraybackslash}X
    | >{\centering\arraybackslash}X 
    |
}
    \hline
    \centering\textbf{\large Component} & \textbf{\large Value} & \textbf{\large Quantity} \HSPC \\
    \hline
    Arduino & UNO & 1 \RSPC \\
    \hline
    Breadboard & - & 1 \RSPC \\
    \hline
    LED & - & 1 \RSPC \\
    \hline
    Jumper Wires & M-M & 10 \RSPC \\
    \hline
    Resistor & 220 $\Omega$ & 1 \RSPC \\
    \hline
\end{tabularx}
\caption{Table of Components}
\label{table:1}
\end{table}
\bigskip

\newpage
\section{Solution}
\subsection{Karnaugh Map}
\begingroup
\large
\begin{math}
\label{eq:F_q}
    \mathtt{
        F(P,Q,R,S) = \sum(0,2,5,7,8,10,13,15) + d(2,7,8,13)
        }
\end{math}
\medskip
\centering
\begin{karnaugh-map}[4][4][1][$Q$][$P$][$S$][$R$]

% Filling K-Map
\terms{2,7,8,13}{X} % don't care terms
\minterms{0,2,5,7,8,10,13,15} % minterm eq
\autoterms[0] % set unfilled to 0

% K-Map groups
\implicant{5}{15}
\implicantcorner

\end{karnaugh-map}\par
\large\raggedright The above K-Map can be simplified to:
\begin{equation}
    \label{eq:Fsimple}
    \LARGE
    \mathtt{
        F = \overline{Q} \cdot \overline{S} + Q \cdot S
        }
\end{equation}
\par\large\raggedright
Therefore, the answer is \hyperref[sec:ques]{\textbf{(A)}}

\endgroup
\bigskip

% Truth Table obtained from K-Map simplification
\subsection{Truth Table}
\label{sec:ttf}
\begin{table}[ht]
\centering
\begin{tabularx}{1\textwidth}{
| >{\centering\arraybackslash}X
| >{\centering\arraybackslash}X
| >{\centering\arraybackslash}X
| >{\centering\arraybackslash}X
| >{\centering\arraybackslash}X |
}
\hline
\HSPC
\large
    \textbf{P} & \large\textbf{Q} & \large\textbf{R} & \large\textbf{S} & \large\textbf{F} \\
\hline
     0&0&0&0&1 \RSPC \\
     0&0&0&1&0 \RSPC \\
     0&0&1&0&1 \RSPC \\
     0&0&1&1&0 \RSPC \\
     0&1&0&0&0 \RSPC \\
     0&1&0&1&1 \RSPC \\
     0&1&1&0&0 \RSPC \\
     0&1&1&1&1 \RSPC \\
     1&0&0&0&1 \RSPC \\
     1&0&0&1&0 \RSPC \\
     1&0&1&0&1 \RSPC \\
     1&0&1&1&0 \RSPC \\
     1&1&0&0&0 \RSPC \\
     1&1&0&1&1 \RSPC \\
     1&1&1&0&0 \RSPC \\
     1&1&1&1&1 \RSPC \\
\hline
\end{tabularx}
\caption{Truth Table for \hyperref[eq:Fsimple]{F}}
\label{tab:ttf}
\end{table}
\bigskip

% Connections
\newpage
\section{Connections}
    \subsection{LED to Arduino}
        LED connections to Arduino are as follows:
        \begin{table}[ht]
            \centering
            \begin{tabular}{|c |c |c |c |}
            \hline
                \textbf{Arduino} & 2 & GND \HSPC \\
            \hline
                \textbf{LED} & + & - \HSPC \\ 
            \hline
            \end{tabular}
            \caption{LED Connections}
            \label{tab:cnct_led}
        \end{table}
        
    \subsection{Input Pins to Arduino} {
        Input Pin Connections to Arduino are as follows:
        \begin{table}[ht]
            \centering
            \begin{tabular}{|c |c |c |c |c |}
            \hline
                \textbf{Arduino} & D6 & D7 & D8 & D9 \HSPC \\
            \hline
                \textbf{Term} & P & Q & R & S \HSPC \\
            \hline
            \end{tabular}
            \caption{Input Pin Connections}
            \label{tab:cnct_ip}
        \end{table}
    }
    
\subsection{Setting Input Pin Values}
        The values of the Input pins are taken by connecting them to either $5V$ or GND according to \hyperref[sec:ttf]{Truth Table}
\bigskip    

\newpage
\section{Code}
\subsection{main.cpp}
The Arduino implementation uses the following code:
% reading code from main.cpp stored in ide/latex/src
\begin{center}
    \lstinputlisting[
    language=C++,
    style=myStyle
    ]{../codes/src/main.cpp}
\end{center}

% Link to Github repository
\subsection{Repository}
Code is also available online at the following repository:
\begin{center}
    \setlength{\fboxsep}{1em}
    \fbox{
        \parbox{0.9\textwidth}{
            \centering
            \small
            \textbf{
                \url{https://github.com/gaurav5-5/ard-workshop/tree/main/ide/latex/codes}
            }
        }
    }
    
\end{center}

\end{document}
